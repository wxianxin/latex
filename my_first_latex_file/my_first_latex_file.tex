\documentclass[12pt]{article}
%\usepackage[left=12mm, right=12mm, top=0.5in, bottom=12mm]{geometry}

\begin{document}
\title{my first \LaTeX \ file}
\author{Steven Wang}
\date{\today}
\maketitle

\tableofcontents
%Need to be built twice

\section{Introduction}
This is the sample intro.\\
This is the 2nd line.\\
This is the 3rd line.\\
%"\\" create different lines; "soft return"


\section{Math}

\subsection{Math Mode}
A rectangular with side lenth $(x+1)$ and $(x+3)$
% $(.)$ turn into math

% $ inline math mode
% $$ 
Take the quation $ax^2+bx+c=0$ as an example.\\
Take the quation $$ax^2+bx+c=0$$ as an example.

\subsection{Common Math Notation}
superscripts:
$$2x^34$$
$$2x^{34}$$
$$2x^{3x+4}$$
$$2x^{3x^4+5}$$

subscripts:
$$x_1$$
$$x_{12}$$
$${x_1}_2$$

greek letters:
$$\pi$$
$$A=\pi r^2$$

trig functions:
$$\sin{x}$$
$$\tan{x}$$
$$y=\sin{x}$$
%You could type in plain text mode. But put it in tex mode, it italicize the variable and space everything properly.

log functions:
$$\log{x}$$
$$\ln{x}$$
$$\log_5{x}$$

square roots:
$$\sqrt{x}$$
$$\sqrt[3]{x}$$
$$\sqrt[4]{x+\sqrt[3]{y}}$$

fractions:\\
About 2/3 of the glass is full.\\
About $\frac{2}{3}$ of the glass is full.\\
About $\displaystyle{\frac{2}{3}}$ of the glass is full.

$$\frac{x}{x^2+x+1}$$
$$\frac{1}{1+\frac{1}{x}}$$


\subsection{Equations}

\begin{equation}
\alpha=\beta
\end{equation}

\section{Brackets, Tables and Arrays}

brackets:

$$(x+1)$$
$$[x+1]$$
$$\{x+1\}$$
$$\$12.55$$

$$3(\frac{2}{3})$$
$$3\left(\frac{2}{3}\right)$$
$$3\left[\frac{2}{3}\right]$$
$$3\left\{\frac{2}{3}\right\}$$

Abosolute value:
$$|x|$$
$$|\frac{x}{y}|$$
$$\left|\frac{x}{y}\right|$$

$$\left\{x^2\right\}$$
$$\left\{x^2\right.$$
%If you only need half of the brackets
$$\left.\frac{dy}{dx}\right|_{x=1}$$
%derivatives

Tables:
\begin{tabular}{|c|ccccc|}
%use the pipe to insert vertical bar
%6 columns with centered entry

$x$ & 1 & 2 & 3 & 4 & 5 \\ \hline
%horizontal line, after the role
$f(x)$ & 10 & 11 & 12 & 13 & 14 \\

\end{tabular}

Table borders:
\begin{tabular}{|c|ccccc|}
\hline
$x$ & 1 & 2 & 3 & 4 & 5 \\ \hline
$f(x)$ & 10 & 11 & 12 & 13 & 14 \\
\hline
\end{tabular}

Array:

\begin{eqnarray}
%automatically in math mode, we do not need to use "$" sign
5x^2-9=x+3\\
5x^2-12=x\\
x^2=3\\
x\approx\pm1.732
\end{eqnarray}

\begin{eqnarray}
%automatically in math mode, we do not need to use "$" sign
5x^2-9&=&x+3\\
5x^2-12&=&x\\
x^2&=&3\\
x&\approx&\pm1.732
\end{eqnarray}

\section{Lists}

\begin{enumerate}
\item pencil
\item ruler
\item eraser
\item calculator
\item notebook
	\begin{enumerate}
	\item assessments
		\begin{enumerate}
		\item tests
		\item quizzes
		\end{enumerate}
	\item homework
	\item notes
	\end{enumerate}
\item paper
\end{enumerate}

\begin{itemize}
\item pencil
\item ruler
\item eraser
\item calculator
\item notebook
	\begin{itemize}
	\item assessments
		\begin{itemize}
		\item tests
		\item quizzes
		\end{itemize}
	\item homework
	\item notes
	\end{itemize}
\item paper
\end{itemize}

\begin{enumerate}
\item[Commutative] $a+b=b+a$
\item[Associative] $(a+b)+c=a+(b+c)$
\item[Distributive] $a(b+c)=ab+ac$
\end{enumerate}

\section{Text Formatting}

\subsection{Fonts}

This will produce \textit{italicized} text.

This will produce \textbf{bold-faced} text.

This will produce \textsc{small caps} text.

This will produce \texttt{teletype} font.
%texttt is particularly good for URL

Please visit \texttt{stevenwang.ddns.net}

Please accept my \begin{large} apology \end{large}.

Please accept my \begin{huge} apology \end{huge}.

Please accept my \begin{small} apology \end{small}.

Please accept my \begin{tiny} apology \end{tiny}.

\subsection{Justification}

\begin{center}This is centered.\end{center}

\begin{flushleft}This is left-justified.\end{flushleft}
%this does not indent
\begin{flushright}This is left-justified.\end{flushright}

\end{document}